% Этот шаблон документа разработан в 2014 году
% Данилом Фёдоровых (danil@fedorovykh.ru) 
% для использования в курсе 
% <<Документы и презентации в \LaTeX>>, записанном НИУ ВШЭ
% для Coursera.org: http://coursera.org/course/latex .
% Исходная версия шаблона --- 
% https://www.writelatex.com/coursera/latex/5.1

\documentclass[c,12pt]{beamer}  % [t], [c], или [b] --- вертикальное выравнивание на слайдах (верх, центр, низ)
%\documentclass[handout]{beamer} % Раздаточный материал (на слайдах всё сразу)
%\documentclass[aspectratio=169]{beamer} % Соотношение сторон

%\usetheme{Berkeley} % Тема оформления
%\usetheme{Bergen}
%\usetheme{Szeged}

%\usecolortheme{beaver} % Цветовая схема
%\useinnertheme{circles}
%\useinnertheme{rectangles}

\usetheme{Madrid}
\usecolortheme{crane}
%%% Работа с русским языком
\usepackage{cmap}					% поиск в PDF
\usepackage{mathtext} 				% русские буквы в формулах
\usepackage[T2A]{fontenc}			% кодировка
\usepackage[utf8]{inputenc}			% кодировка исходного текста
\usepackage[english,russian]{babel}	% локализация и переносы

%% Beamer по-русски
\newtheorem{rtheorem}{Теорема}
\newtheorem{rproof}{Доказательство}
\newtheorem{rexample}{Пример}

%%% Дополнительная работа с математикой
\usepackage{amsmath,amsfonts,amssymb,amsthm,mathtools} % AMS
\usepackage{icomma} % "Умная" запятая: $0,2$ --- число, $0, 2$ --- перечисление

%% Номера формул
%\mathtoolsset{showonlyrefs=true} % Показывать номера только у тех формул, на которые есть \eqref{} в тексте.
%\usepackage{leqno} % Нумерация формул слева

%% Свои команды
\DeclareMathOperator{\sgn}{\mathop{sgn}}

%% Перенос знаков в формулах (по Львовскому)
\newcommand*{\hm}[1]{#1\nobreak\discretionary{}
	{\hbox{$\mathsurround=0pt #1$}}{}}

%%% Работа с картинками
\usepackage{graphicx}  % Для вставки рисунков
\graphicspath{{images/}{images2/}}  % папки с картинками
\setlength\fboxsep{3pt} % Отступ рамки \fbox{} от рисунка
\setlength\fboxrule{1pt} % Толщина линий рамки \fbox{}
\usepackage{wrapfig} % Обтекание рисунков текстом

%%% Работа с таблицами
\usepackage{array,tabularx,tabulary,booktabs} % Дополнительная работа с таблицами
\usepackage{longtable}  % Длинные таблицы
\usepackage{multirow} % Слияние строк в таблице

%%% Программирование
\usepackage{etoolbox} % логические операторы

%%% Другие пакеты
\usepackage{lastpage} % Узнать, сколько всего страниц в документе.
\usepackage{soul} % Модификаторы начертания
\usepackage{csquotes} % Еще инструменты для ссылок
%\usepackage[style=authoryear,maxcitenames=2,backend=biber,sorting=nty]{biblatex}
\usepackage{multicol} % Несколько колонок

%%% Картинки
\usepackage{tikz} % Работа с графикой
\usepackage{pgfplots}
\usepackage{pgfplotstable}

\usepackage{tcolorbox}
\usepackage{textcomp}
\renewcommand{\theenumi}{\asbuk{enumi}}
%\DeclareMathOperator{\tg}{tg}
\newcommand{\tgx}{\tg x}

\title{Геометрия треугольника}
\subtitle{ }
\author{\textcopyright}
\date{}
\begin{document}
	
	\frame[plain]{\titlepage}	% Титульный слайд
	
	\section{}
	\subsection{}
	\begin{frame}
		\begin{block}{}
			В прямоугольном треугольнике $ ABC $ точка $ M $ лежит на катете $ AC $, а точка $ N $ лежит на продолжении катета $ BC $ за точку $ C $, причём и Отрезки $ CP $ и $ CQ $ — биссектрисы треугольников $ ACB $ и $ NCM $ соответственно.
			
			а) Докажите, что $ CP $ и $ СQ $ перпендикулярны.
			
			б) Найдите $ PQ $, если $ BC=3 $, а $ AC=5 $.
			
			(ЕГЭ-2019, основная волна)
		\end{block}
	\end{frame}
	\begin{frame}
		\begin{block}{}
			В трапеции $ABCD$ основание $AD = 2BC$. Точка $M$ внутри
			трапеции такова, что углы $ABM$ и $MCD$ прямые.
			
			а) Докажите, что $MA = MD$.
			
			б) Найдите угол $BAD$, если расстояние от точки $M$ до $AD$ равно
			$BC$, а угол $ADC$ равен $55^{\circ}$.
			
			(ЕГЭ-2017)
		\end{block}
	\end{frame}

	\begin{frame}
		\begin{block}{}
			В прямоугольном треугольнике $ABC$ точки $M$ и $N$ – середины
			гипотенузы $AB$ и катета $BC$ соответственно. Биссектриса угла
			BAC пересекает прямую $ MN $ в точке $L$.
			
			а) Докажите, что треугольники $ AML $ и $ BLC $ подобны.
			
			б) Найдите отношение площадей этих треугольников, если $\cos \angle BAC = \dfrac{7}{25} $.
			
			(ЕГЭ-2016)
		\end{block}
	\end{frame}


	\begin{frame}
		\begin{block}{}
			В прямоугольном треугольнике $ ABC $ с прямым углом $ C $ точки $ M $ и $ N $ – середины катетов $ AC $ и $ BC $ соответственно, $ CH $ – высота.
			
			а) Докажите, что $ MH \perp NH $ .
			
			б) Пусть $ P $ – точка пересечения прямых $ AC $ и $ NH $, а $ Q $ – точка	пересечения прямых $ BC $ и $ MH $. Найдите площадь треугольника
			$ PQM $, если $ AH = 4 $ и $ BH = 2. $
			
			(ЕГЭ-2016)
		\end{block}
	\end{frame}

	\begin{frame}
		\begin{block}{}
			В треугольнике $ ABC $ проведены высоты $ AK $ и $ CM $. На них из	точек $ M $ и $ K $ опущены перпендикуляры ME и KH соответственно.
			
			а) Докажите, что $ EH || AC $ ;
			
			б) Найдите отношение $ EH : AC $, если $\angle ABC= 30^\circ $.
			
			(ЕГЭ-2016)
		\end{block}
	\end{frame}

	\title{Геометрия четырехугольника}
	\frame[plain]{\titlepage}	% Титульный слайд
	
	\begin{frame}
		\begin{block}{}
			Дан параллелограмм $ ABCD $. Точка $ М $ лежит на диагонали $ BD $ и делит ее в отношении $ 1 : 2 $. Найдите площадь параллелограмма $ ABCD $, если площадь четырехугольника $ ABCM $ равна 60. В ответе запишите максимальную площадь параллелограмма $ ABCD $.
		\end{block}
		\begin{block}{}
			В трапеции $ ABCD $ основание $ AD $ равно 16, сумма диагоналей $ AC $ и $ BD $ равна 36, $\angle CAD = 60^\circ $. Отношение площадей треугольников $ AOD $ и $ BOC $, где $ O $ — точка пересечения диагоналей, равно 4. Найдите площадь трапеции.
		\end{block}
		\begin{block}{}
			Диагонали равнобокой трапеции взаимно перпендикулярны. Длины оснований равны 2 и 5. Найдите длину высоты трапеции.
		\end{block}
	\end{frame}
	\begin{frame}
		\begin{block}{}
			Дан треугольник $ ABC $. На сторонах $ AB$, BC $ и $ CA $ взяты точки $ X$, Y $ и $ Z $ соответственно таким образом, что четырёхугольник $ XYCZ $ является ромбом. Известно, что $ AZ =1 $, а $ BY = 4 $. Найдите сторону ромба.
		\end{block}		
		\begin{block}{}
			Средняя линия трапеции равна 5, а отрезок, соединяющий середины оснований, равен 3.			
			Углы при большем основании трапеции равны $ 30^\circ $ и $ 60^\circ $. Найдите сумму длин большего основания и меньшей боковой стороны трапеции.
		\end{block}
		
		\begin{block}{}
			Диагональ $ AC $ прямоугольника $ ABCD $ с центром $ О $ образует со стороной $ AB $ угол $ 30^\circ $. Точка $ Е $ лежит вне прямоугольника, причем $ \angle BEC = 120^\circ$.
			
			а)	Докажите, что $\angle CBE = \angle COE$.
			
			б)	Прямая $ OE $ пересекает сторону $ AD $ прямоугольника в точке $ K $. Найдите $ EK $, если известно, что $  BE = 40 $ и $ CE = 24 $.
		\end{block}		
	\end{frame}

	\begin{frame}
		\begin{block}{}		
			На диагонали параллелограмма взяли точку, отличную от ее середины. Из нее на все стороны параллелограмма (или их продолжения) опустили перпендикуляры.
			
			а)	Докажите, что четырехугольник, образованный основаниями этих перпендикуляров, является трапецией.
			
			б)	Найдите площадь полученной трапеции, если площадь параллелограмма равна 16, а один из его углов равен $ 60^\circ $.			
		\end{block}
		\begin{block}{}
			Дан четырехугольник $ ABCD $.
			
			а)	Докажите, что отрезки $ LN $ и $ KM $ соединяющие середины его противоположных сторон, делят друг друга пополам.
			
			б)	Найдите площадь четырехугольника $ ABCD $, если $ LM = 3 \sqrt{3}, KM = 6 \sqrt{3}, \angle KML = 60^\circ$.
		\end{block}
	\end{frame}
	\begin{frame}
		\begin{block}{}
			На сторонах $ AD $ и $ BC $ параллелограмма $ ABCD  $  соответственно точки $ M $ и $ N $, причем $ M $ - середина $AD$, а $ BN : NC = 1:3 $.
			
			а)	Докажите, что прямые $ AN $ и $ AC $ делят отрезок $ BM $ на три равные части.
			
			б)	Найдите площадь четырехугольника, вершины которого находятся в точках $ С, N $ и точках пересечения прямой $ BM $ с прямыми $ AN $ и $ AC $, если  $ S_{ABCD} = 48$.
		\end{block}
	\end{frame}
	\begin{frame}
		\begin{block}{}
			Дана равнобедренная трапеция, в которой $AD = 3BC, CM$–
			высота трапеции.
			
			а) Докажите, что $M$ делит $AD$ в отношении 2 : 1.
			
			б) Найдите расстояние от точки $C$ до середины $BD$, если
			$AD = 18, AC = 4 \sqrt{13}$.
			
			(ЕГЭ-2017)
		\end{block}		
	\end{frame}

	\begin{frame}
		\begin{block}{}
			Дана трапеция с диагоналями равными 8 и 15. Сумма длин
			оснований равна 17.
			
			а) Докажите, что диагонали перпендикулярны.
			
			б) Найдите площадь трапеции.
			
			(ЕГЭ-2017)
		\end{block}
	\end{frame}

	\begin{frame}
		\begin{block}{}
			Точка $E$ – середина боковой стороны $CD$ трапеции $ABCD$. На
			стороне $AB$ взяли точку $K$ так, что прямые $CK$ и $AE$ параллельны. Отрезоки $CK$ и $BE$ пересекаются в точке $O$.
			
			а) Докажите, что $CO = KO$.
			
			б) Найти отношение длин оснований трапеции $BC$ и $AD$, если
			площадь треугольника $BCK$ составляет $\frac{9}{64}$ от площади трапеции
		 $ABCD$.
			(ЕГЭ-2017)
		\end{block}
	\end{frame}

	\begin{frame}
		\begin{block}{}
			Точки $E$ и $K$ – середины сторон $CD$ и $AD$ квадрата $ABCD$ соответственно. Прямая $BE$ пересекается с прямой $CK$ в точке $O$.
			
			а) Докажите, что вокруг четырёхугольника $ABOK$ можно
			описать окружность.
			
			б) Найдите $AO$, если сторона квадрата равна 1.
			
			(ЕГЭ-2017)
		\end{block}
	\end{frame}

	\begin{frame}
		\begin{block}{}
			Прямая, проходящая через вершину $B$, прямоугольника $ABCD$,
			перпендикулярная диагонали $AC$ и пересекает сторону $AD$ в
			точке $M$, равноудаленной от вершин $B$ и $D$.
			
			а) Докажите, что $BM$ и $BD$ делят угол $B$ на три равных угла.
			
			б) Найдите расстояние от точки пересечения диагоналей
			прямоугольника $ABCD$ до прямой $CM$, если $BC = 6 \sqrt{21}$.
			
			(ЕГЭ-2016)
		\end{block}
	\end{frame}

	\begin{frame}
		\begin{block}{}
			Точка $M лежит на стороне $BC выпуклого четырёхугольника
		 $ABCD$, причём $B и $C – вершины равнобедренных треугольников
			с основаниями $AM и $DM соответственно, а прямые $AM и $MD
			перпендикулярны.
			
			а) Докажите, что биссектрисы углов при вершинах $B и $C
			четырёхугольника $ABCD$, пересекаются на стороне $AD$.
			
			б) Пусть $N$ – точка пересечения этих биссектрис. Найдите
			площадь четырёхугольника $ABCD$, если известно, что
		 $BM : MC = 3 : 4$, а площадь четырёхугольника, стороны
			которого лежат на прямых $AM$, DM$, BN$ и $CN$, равна 24.
			
			(ЕГЭ-2015)
		\end{block}
	\end{frame}

\title{Геометрия окружности}
\frame[plain]{\titlepage}	% Титульный слайд

	\begin{frame}
		\begin{block}{}
			Около треугольника $ ABC $ описана окружность. Прямая $ BO $, где $ O $ — центр вписанной окружности, вторично пересекает описанную окружность в точке $ P $.
			
			а) Докажите, что $ OP=AP $.
			
			б) Найдите расстояние от точки $ P $ до прямой $ AC $, если $ \angle ABC $ а радиус описанной окружности равен 18.
			
			(ЕГЭ-2019, основная волна)
		\end{block}
	\end{frame}

	\begin{frame}
		\begin{block}{}
			Около остроугольного треугольника $ ABC $ с различными сторонами описали окружность с диаметром $ BN $. Высота $ BH $ пересекает эту окружность в точке $ K $.
			
			а) Докажите, что $ AN=CK $
			
			б) Найдите $ KN $, если $ \angle BAC=35^\circ, \angle ACB=65^\circ $, а радиус окружности равен 12.
			
			(ЕГЭ-2019, основная волна)
		\end{block}
	\end{frame}
	\begin{frame}
		\begin{block}{}
			Точка $ O $ — центр вписанной в треугольник $ ABC $ окружности. Прямая $ OB $ вторично пересекает описанную около этого треугольника окружность в точке $ P $.
			
			а) Докажите, что $ \angle POC=\angle PCO $.
			
			б) Найдите площадь треугольника $ APC $, если радиус описанной около треугольника $ ABC $ окружности равен 4, а $ \angle ABC=120^\circ $
			
			(ЕГЭ-2019, основная волна)
		\end{block}
	\end{frame}

	\begin{frame}
		\begin{block}{}
			В остроугольном треугольнике $ ABC $, Высоты $ BN $ и $ CM $ треугольника $ ABC $ пересекаются в точке $ H $. Точка $ O $ — центр окружности, описанной около треугольника $ ABC $
			
			а) Докажите, что $ AH=AO $.
			
			б) Найдите площадь треугольника $ AHO $, если $ BC=6\sqrt{3}$ , $ \angle ABC=45^\circ $.
			
			(ЕГЭ-2019, основная волна)
		\end{block}
	\end{frame}
	
	\begin{frame}
		\begin{block}{}
			Окружность касается стороны $ AC $ остроугольного треугольника $ ABC $ и делит каждую из сторон $ AB $ и $ BC $ на три равные части.
			
			а) Докажите, что $ AB=BC $.
			
			б) Найдите, в каком отношении высота этого треугольника, проведённая из вершины $ A $, делит сторону $ BC $.
			
			(ЕГЭ-2019, резервный день)
		\end{block}
	\end{frame}

	\begin{frame}
		\begin{block}{}
			Из вершины $ С $ прямого угла прямоугольного треугольника $ ABC $ проведена высота $ CH $.
			
			а) Докажите, что отношение площадей кругов, построенных на отрезках $ AH $ и $ BH $ соответственно как на диаметрах равно $ \tg^4 \angle ABC $
			
			б) Пусть точка $ O_1 $ — центр окружности диаметра $ AH $, вторично пересекающей отрезок $ AC $ в точке $ P $, а точка $ O_2 $ — центр окружности с диаметром $ BH $, вторично пересекающей отрезок $ BC $ в точке $ Q $. Найдите площадь четырёхугольника $ O_1PQO_2 $, если $ AC=22, BC=18 $
			
			(ЕГЭ-2019, резервный день)
		\end{block}
	\end{frame}
	\begin{frame}
		\begin{block}{}
			Две окружности касаются внешним образом в точке $K$. Прямая
		 $AB$ касается первой окружности в точке $A$, а второй – в точке $B$.
			Прямая $BK$ пересекает первую окружность в точке $D$, прямая
		 $AK$ пересекает вторую окружность в точке $C$.
			
			а) Докажите, что прямые $AD$ и $BC$ параллельны.
			
			б) Найдите площадь треугольника $AKB$, если известно, что
			радиусы окружностей равны 4 и 1.
			
			(ДемоЕГЭ-2018)
		\end{block}
	\end{frame}
	

	\begin{frame}
		\begin{block}{}
			Две окружности с центрами $O_1$ и $O_2$ пересекаются в точках $A$ и
		 $B$, причем точки $O_1$ и $O_2$ лежат по разные стороны от прямой
		 $AB$. Продолжение диаметра $CA$ первой окружности и хорды $CB$
			этой же окружности пересекают вторую окружность в точках $D$ и
		 $E$ соответственно.
			
			а) Докажите, что треугольники $CBD$ и $O_1 A O_2$ подобны.
			
			б) Найдите $AD$, если угол $DAE$ равен углу $BAC$, а радиус второй
			окружности в четыре раза больше радиуса первой и $AB = 2$.
			
			(ЕГЭ-2017)
		\end{block}
	\end{frame}

	\begin{frame}
		\begin{block}{}
			Две окружности с центрами $ O_1$ и $ O_2$ и радиусами 3 и 4
			пересекаются в точках $A$ и $B$. Через точку $A$ проведена прямая
		 $MK$ пересекающая обе окружности в точках $M$ и $K$, причем
			точка $A$ находится между ними.
			
			а) Докажите, что треугольники $BMK$ и $O_1 A O_2$ подобны.
			
			б) Найдите расстояние от точки $B$ до прямой $MK$, если
			$O_1 O_2$ = 5, $MK = 7$.
			
			(ЕГЭ-2017)
		\end{block}
	\end{frame}

	\begin{frame}
		\begin{block}{}
			Две окружности касаются внутренним образом в точке $A$, причем
			меньшая окружность проходит через через центр $O$ большей.
			Диаметр $BC$ большей окружности вторично пересекает меньшую
			окружность в точке $M$, отличной от $A$. Лучи $AO$ и $AM$ вторично
			пересекают большую окружность в точках $P$ и $Q$ соответственно.
			Точка $C$ лежит на дуге $AQ$ большей окружности, не содержащей
			точку $P$.
			
			а) Докажите, что прямые $PQ$ и $BC$ параллельны.
			
			б) Известно, что $ \sin \angle AOC = \frac{\sqrt5}{3}$, прямые $PC и $AQ
			пересекаются в точке $K$. Найдите отношение $QK : KA$.
			
			(ЕГЭ-2017)
		\end{block}
	\end{frame}

	\begin{frame}
		\begin{block}{}
			В прямоугольном треугольнике $ABC$ проведена высота $CH$ из
			вершины прямого угла. В треугольники $ACH$ и $BCH$ вписаны
			окружности с центрами $O_1$ и $O_2$ соответственно, касающиеся
			прямой $CH$ в точках $M$ и $N$ соответственно.
			
			а) Докажите, что прямые $AO_1$ и $CO_2$ перпендикулярны.
			
			б) Найдите площадь четырёхугольника $MO_1 NO_2$ , если $AC = 20$ и
		 $BC = 15$.
			
			(ЕГЭ-2017)
		\end{block}
	\end{frame}

	\begin{frame}
		\begin{block}{}
			В трапеции $ABCD$ угол $BAD$ прямой. Окружность, построенная
			на большем основании $AD$ как на диаметре, пересекает меньшее
			основание $BC$ в точке $C$ и $M$.
			
			а) Докажите, что угол $BAM$ равен углу $CAD$.
			
			б) Диагонали трапеции $ABCD$ пересекаются в точке $O$. Найдите
			площадь треугольника $AOB$, если $AB = 6$, а $BC = 4BM$.
			
			(ЕГЭ-2017)
		\end{block}
	\end{frame}

\title{Задание 16}
\frame[plain]{\titlepage}	% Титульный слайд

	\begin{frame}
		\begin{block}{}
			Окружность, вписанная в трапецию $ABCD$, касается ее боковых
			сторон $AB$ и $CD$ в точках $M$ и $N$ соответственно. Известно, что
		 $AM = 8MB$ и $DN = 2CN$.
			
			а) Докажите, что $AD = 4BC$.
			
			б) Найдите длину отрезка $MN$, если радиус окружности
			равен $\sqrt6$.
			
			(ЕГЭ-2017)
		\end{block}
	\end{frame}

	\begin{frame}
		\begin{block}{}
			В трапецию $ABCD$ с основаниями $AD$ и $BC$ вписана окружность
			с центром $O$.
			
			а) Докажите, что $ \sin \angle AOD = \sin \angle BOC$.
			
			б) Найдите площадь трапеции, если $\angle BAD = 90^{\circ}$  , а основания
			равны 5 и 7.
			
			(ЕГЭ-2017)
		\end{block}
	\end{frame}

	\begin{frame}
		\begin{block}{}
			В треугольнике $ABC$ угол $ABC$ равен $60^\circ$. Окружность,
			вписанная в треугольник, касается стороны $AC$ в точке $M$.
			
			а) Докажите, что отрезок $BM$ не больше утроенного радиуса
			вписанной в треугольник окружности.
			
			б) Найдите $ \sin  \angle BMC$ если известно, что отрезок $BM$ в 2,5 раза
			больше радиуса вписанной в треугольник окружности.
			
			(ЕГЭ-2016)
		\end{block}
	\end{frame}

	\begin{frame}
		\begin{block}{}
			Квадрат $ABCD$ вписан в окружность. Хорда $CE$ пересекает его
			диагональ $BD$ в точке $K$.
			
			а) Докажите, что  $CK \cdot CE = AB \cdot CD$ .
			
			б) Найдите отношение $CK$ и $KE$, если $ \angle ECD = 15^\circ$.
			
			(ЕГЭ-2016)
		\end{block}
	\end{frame}

	\begin{frame}
		\begin{block}{}
			Окружность касается стороны $AC$ остроугольного треугольника
		 $ABC$ и делит каждую из сторон $AB$ и $BC$ на три равные части.
			
			а) Докажите, что треугольник $ABC$ равнобедренный.
			
			б) Найдите, в каком отношении высота этого треугольника делит
			сторону $BC$.
			
			(ЕГЭ-2016)
		\end{block}
	\end{frame}

	\begin{frame}
		\begin{block}{}
			Окружность, построенная на медиане $BM$ равнобедренного
			треугольника $ABC$ как на диаметре, второй раз пересекает
			основание $BC$ в точке $K$.
			
			а) Докажите, что отрезок $BK$ больше отрезка $CK$.
			
			б) Пусть указанная окружность пересекает сторону $AB$ в точке
		 $N$. Найдите $AB$, если $BK$ = 18 и $BN$ = 17.
			
			(ЕГЭ-2015)
		\end{block}
	\end{frame}
\end{document}